\documentclass[8pt, aspectratio=43]{beamer}

\usepackage{HPE_Lecture}
%\usetikzlibrary{arrows.meta} %In HPE_Lecture.sty eingefügt! (Tikz - Bilder in LaTeX)
\usepackage{ifthen} %usepackage um eine if-Verzweigung zu benutzen
\usepackage{tikz}
\usetikzlibrary{angles,quotes, calc, positioning}

% % % % % % % % % % % % % % % % % % % % % % % % % % %

%Dokument mit allen Ziegerdiagrammen, die im Skript für NuS II enthalten sind. 

% % % % % % % % % % % % % % % % % % % % % % % % % % %

\begin{document}
	

\begin{frame}\frametitle{Zeigerdiagramm III}

	\tikzsetnextfilename{Zeigerdiagramm_III_a}
	\begin{tikzpicture}
		[Spannungspfeil/.style={-{Latex[length=2mm]},thick, blue},
		 Strompfeil/.style={-{Latex[length=2mm]},thick, red},
		 Winkelpfeil/.style={pic text=$\varphi$, draw, -{Latex[length=1mm]}, angle radius = 7mm, angle eccentricity=1.2}
		]

		\def\USpitze{1.5cm};
		\def\ISpitze{1cm};
		\def\Winkel{30};
		\def\Xstep{0.3};
		
		\coordinate (aa) at (-\Winkel:\ISpitze);
		\coordinate (ba) at (0:0cm);
		\coordinate (ca) at (0:\USpitze);
		%\coordinate (Bb) at ($(Ca) + \Xstep*(0:1cm)$);
		
		\draw[Spannungspfeil] (ba) -- (ca) node[midway, above] {$\mzeiger{u}$};
		\draw[Strompfeil] (ba) -- (aa) node[midway, below] {$\mzeiger{i}$};
		\path (aa) -- (ba) -- (ca) pic[Winkelpfeil] {angle = aa--ba--ca};
			
	\end{tikzpicture}
	
\end{frame}
\end{document}