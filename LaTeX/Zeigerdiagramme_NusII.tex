\documentclass[8pt, aspectratio=43]{beamer}

\usepackage{HPE_Lecture}
%\usetikzlibrary{arrows.meta} %In HPE_Lecture.sty eingefügt! (Tikz - Bilder in LaTeX)
\usepackage{ifthen} %usepackage um eine if-Verzweigung zu benutzen
\usepackage{tikz}
\usetikzlibrary{angles,quotes}

% % % % % % % % % % % % % % % % % % % % % % % % % % %

%Dokument mit allen Ziegerdiagrammen, die im Skript für NuS II enthalten sind. 

% % % % % % % % % % % % % % % % % % % % % % % % % % %

\begin{document}
	
%definieren von styles
%\tikzset{FarbeStrom/.style=red}
%\tikzset{FarbeSpannung/.style=blue}
		
%%%%%%%%%%%%%%%%%%%%%%%%%%%%%%%%%%%%%%%%%%%%%%%%%%%%%%%%%%%%%%%%%%%%%%%%%%%

\begin{frame}\frametitle{Zeigerdiagramm I}

%		\tikzsetnextfilename{Zeigerdiagramm_I_a}
%		\begin{tikzpicture}
%		
%		\end{tikzpicture}

\end{frame}

%%%%%%%%%%%%%%%%%%%%%%%%%%%%%%%%%%%%%%%%%%%%%%%%%%%%%%%%%%%%%%%%%%%%%%%%%%%

\begin{frame}\frametitle{Zeigerdiagramm II}



%		\tikzsetnextfilename{Zeigerdiagramm_RLC_Serie_funter_Schleife}
%		\begin{tikzpicture}
%	
%		\end{tikzpicture}


\end{frame}

%%%%%%%%%%%%%%%%%%%%%%%%%%%%%%%%%%%%%%%%%%%%%%%%%%%%%%%%%%%%%%%%%%%%%%%%%%%

\begin{frame}\frametitle{Zeigerdiagramm III}

	\tikzsetnextfilename{Zeigerdiagramm_III_a}
	\begin{tikzpicture}
		\coordinate (A) at (-30:3cm);
		\coordinate (B) at (0:0cm);
		\coordinate (C) at (0:4cm);
		
		\draw[->] (B) -- (C);
		\draw[->] (B) -- (A);
		\path (A) -- (B) -- (C) pic[draw,->,pic text = $\phi$, angle radius = 9mm, angle eccentricity=1.2] {angle};
		
		
	\end{tikzpicture}
%	
%	\tikzsetnextfilename{Zeigerdiagramm_III_b}
%	\begin{tikzpicture}
%	
%	\end{tikzpicture}
%	
%	\tikzsetnextfilename{Zeigerdiagramm_III_c}§
%	\begin{tikzpicture}
%	
%	\end{tikzpicture}
%	
%	\tikzsetnextfilename{Zeigerdiagramm_III_d}
%	\begin{tikzpicture}
%	
%	\end{tikzpicture}

\end{frame}

%%%%%%%%%%%%%%%%%%%%%%%%%%%%%%%%%%%%%%%%%%%%%%%%%%%%%%%%%%%%%%%%%%%%%%%%%%%%%

\end{document}

