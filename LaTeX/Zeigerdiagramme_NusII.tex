\documentclass[8pt, aspectratio=43]{beamer}

\usepackage{HPE_Lecture}
%\usetikzlibrary{arrows.meta} %In HPE_Lecture.sty eingefügt! (Tikz - Bilder in LaTeX)
\usepackage{ifthen} %usepackage um eine if-Verzweigung zu benutzen
\usepackage{tikz}
\usetikzlibrary{angles,quotes, calc, positioning}

% % % % % % % % % % % % % % % % % % % % % % % % % % %

%Dokument mit allen Ziegerdiagrammen, die im Skript für NuS II enthalten sind. 

% % % % % % % % % % % % % % % % % % % % % % % % % % %

\begin{document}
	
%definieren von styles
%\tikzset{Spannungspfeil/.style={-{Latex[length=2mm]},thick, blue}}
%\tikzset{Strompfeil/.style={-{Latex[length=2mm]},thick, red}}
		
%%%%%%%%%%%%%%%%%%%%%%%%%%%%%%%%%%%%%%%%%%%%%%%%%%%%%%%%%%%%%%%%%%%%%%%%%%%

\begin{frame}\frametitle{Zeigerdiagramm I}

%		\tikzsetnextfilename{Zeigerdiagramm_I_a}
%		\begin{tikzpicture}
%		
%		\end{tikzpicture}

\end{frame}

%%%%%%%%%%%%%%%%%%%%%%%%%%%%%%%%%%%%%%%%%%%%%%%%%%%%%%%%%%%%%%%%%%%%%%%%%%%

\begin{frame}\frametitle{Zeigerdiagramm II}



%		\tikzsetnextfilename{Zeigerdiagramm_RLC_Serie_funter_Schleife}
%		\begin{tikzpicture}
%	
%		\end{tikzpicture}


\end{frame}

%%%%%%%%%%%%%%%%%%%%%%%%%%%%%%%%%%%%%%%%%%%%%%%%%%%%%%%%%%%%%%%%%%%%%%%%%%%

\begin{frame}\frametitle{Zeigerdiagramm III}

	\tikzsetnextfilename{Zeigerdiagramm_III_a}
	\begin{tikzpicture}
		[Spannungspfeil/.style={-{Latex[length=2mm]},thick, blue},
		 Strompfeil/.style={-{Latex[length=2mm]},thick, red},
		 Winkelpfeil/.style={pic text=$\varphi$, draw, -{Latex[length=1mm]}, angle radius = 7mm, angle eccentricity=1.2}
		]

		\def\USpitze{1.5cm};
		\def\ISpitze{1cm};
		\def\Winkel{30}
		\def\Xstep{0.3}
		
		\coordinate (Aa) at (-\Winkel:\ISpitze);
		\coordinate (Ba) at (0:0cm);
		\coordinate (Ca) at (0:\USpitze);
		%\coordinate (Bb) at ($(Ca) + \Xstep*(0:1cm)$);
		
		\draw[Spannungspfeil] (Ba) -- (Ca) node[midway, above] {$\mzeiger{u}$};
		\draw[Strompfeil] (Ba) -- (Aa) node[midway, below] {$\mzeiger{i}$};
		\path (Aa) -- (Ba) -- (Ca) pic[Winkelpfeil] {angle = Aa--Ba--Ca};
			
	\end{tikzpicture}
	
%	\tikzsetnextfilename{Zeigerdiagramm_III_b}
%	\begin{tikzpicture}
%		[Spannungspfeil/.style={-{Latex[length=2mm]},thick, blue},
%		 Strompfeil/.style={-{Latex[length=2mm]},thick, red},
%		 Winkelpfeil/.style={pic text=$\varphi$, draw, -{Latex[length=1mm]}, angle radius = 7mm, angle eccentricity=1.2}
%		]
%				
%		\def\USpitze{1.5cm};
%		\def\ISpitze{1cm};
%		\def\Winkel{30}
%		
%		\coordinate (A) at (0:\ISpitze);
%		\coordinate (B) at (0:0cm);
%		\coordinate (C) at (\Winkel:\USpitze);
%	
%		\draw[Strompfeil] (B) -- (A) node[midway, below] {$\mzeiger{i}$};
%		\draw[Spannungspfeil] (B) -- (C) node[midway, above] {$\mzeiger{u}$};
%		\path (A) -- (B) -- (C) pic[Winkelpfeil] {angle};
%	
%	\end{tikzpicture}
%	
%	\tikzsetnextfilename{Zeigerdiagramm_III_c}§
%	\begin{tikzpicture}
%	
%	\end{tikzpicture}
%	
%	\tikzsetnextfilename{Zeigerdiagramm_III_d}
%	\begin{tikzpicture}
%	
%	\end{tikzpicture}

\end{frame}

%%%%%%%%%%%%%%%%%%%%%%%%%%%%%%%%%%%%%%%%%%%%%%%%%%%%%%%%%%%%%%%%%%%%%%%%%%%%%

\end{document}

