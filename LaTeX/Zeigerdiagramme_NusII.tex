\documentclass[8pt, aspectratio=43]{beamer}

\usepackage{HPE_Lecture}
%\usetikzlibrary{arrows.meta} %In HPE_Lecture.sty eingefügt! (Tikz - Bilder in LaTeX)
\usepackage{ifthen} %usepackage um eine if-Verzweigung zu benutzen
\usepackage{tikz}
\usetikzlibrary{angles, calc}

% % % % % % % % % % % % % % % % % % % % % % % % % % %

%Dokument mit allen Ziegerdiagrammen, die im Skript für NuS II enthalten sind. 

% % % % % % % % % % % % % % % % % % % % % % % % % % %

\begin{document}
	
%%%%%%%%%%%%%%%%%%%%%%%%%%%%%%%%%%%%%%%%%%%%%%%%%%%%%%%%%%%%%%%%%%%%%%%%%%%

\begin{frame}\frametitle{Zeigerdiagramm I}

%		\tikzsetnextfilename{Zeigerdiagramm_I_a}
%		\begin{tikzpicture}
%		
%		\end{tikzpicture}

\end{frame}

%%%%%%%%%%%%%%%%%%%%%%%%%%%%%%%%%%%%%%%%%%%%%%%%%%%%%%%%%%%%%%%%%%%%%%%%%%%

\begin{frame}\frametitle{Zeigerdiagramm II}



%		\tikzsetnextfilename{Zeigerdiagramm_RLC_Serie_funter_Schleife}
%		\begin{tikzpicture}
%	
%		\end{tikzpicture}


\end{frame}

%%%%%%%%%%%%%%%%%%%%%%%%%%%%%%%%%%%%%%%%%%%%%%%%%%%%%%%%%%%%%%%%%%%%%%%%%%%

\begin{frame}\frametitle{Zeigerdiagramm III}

	\tikzsetnextfilename{Zeigerdiagramm_III_a_b_c_d}
	\begin{tikzpicture}
		[Spannungspfeil/.style={-{Latex[length=2mm]},thick, blue},
		 Strompfeil/.style={-{Latex[length=2mm]},thick, red},
		 Winkelpfeil/.style={pic text=$\varphi$, draw, -{Latex[length=1mm]}, angle radius = 7mm, angle eccentricity=1.2},
		 Winkelpfeil2/.style={pic text=$\varphi$, draw, -{Latex[length=1mm]}, angle radius = 4mm, angle eccentricity=1.4},
		 Beschriftung/.style={anchor=north, gray}
		]

		\def\USpitze{1.5cm};
		\def\ISpitze{1cm};
		\def\Winkel{35};
		\def\Xstep{0.3cm};
		\def\BeschSpace{4};
		
		% ---------------------------------------------------------------------- % 
		%% Diagramm a) %%
		% Koordinaten a) %
		\coordinate (Aa) at (-\Winkel:\ISpitze);
		\coordinate (Ba) at (0:0cm);
		\coordinate (Ca) at (0:\USpitze);
		% Pfeile a) % 
		\draw[Spannungspfeil] (Ba) -- (Ca) node[midway, above] {$\mzeiger{u}$};
		\draw[Strompfeil] (Ba) -- (Aa) node[midway, below] {$\mzeiger{i}$};
		\path (Aa) -- (Ba) -- (Ca) pic[Winkelpfeil] {angle = Aa--Ba--Ca};
		% Beschriftung a) %
		\node[Beschriftung] at (-90:{\ISpitze / \BeschSpace}) {a)};
		% ---------------------------------------------------------------------- % 
		%% Diagramm b) %% 
		% Koordinaten b) % 
		\coordinate (Bb) at ($(Ca) + (0:\Xstep)$); 
		\coordinate (Ab) at ($(Bb) + (0:\ISpitze)$);
		\coordinate (Cb) at ($(Bb) + (\Winkel:\USpitze)$);
		% Pfeile b) %
		\draw[Spannungspfeil] (Bb) -- (Cb) node[midway, above] {$\mzeiger{u}$};
		\draw[Strompfeil] (Bb) -- (Ab) node[midway, below] {$\mzeiger{i}$};
		\path (Ab) -- (Bb) -- (Cb) pic[Winkelpfeil] {angle = Ab--Bb--Cb};
		% Beschriftung b) %
		\node[Beschriftung] at ($(Bb) + (-90:{\ISpitze / \BeschSpace})$) {b)};
		% ---------------------------------------------------------------------- % 
		%% Diagramm c) %%
		% Koordinaten c) %
		\coordinate (Bc) at ($(Ab) + (0:\Xstep)$);
		\coordinate (Ac) at ($(Bc) + (0:\ISpitze)$);
		\coordinate (Cc) at ($(Ac) + (180+\Winkel:\USpitze)$);
		% Pfeile c) %
		\draw[Spannungspfeil] (Cc) -- (Ac) node[midway, below] {$\mzeiger{u}$};
		\draw[Strompfeil] (Bc) -- (Ac) node[midway, above] {$\mzeiger{i}$};
		\path (Bc) -- (Ac) -- (Cc) pic[Winkelpfeil] {angle = Bc--Ac--Cc};
		% Beschriftung c) %
		\node[Beschriftung] at ($(Bc) + (-90:{\ISpitze / \BeschSpace})$) {c)};
		% ---------------------------------------------------------------------- % 
		%% Diagramm d) %%
		% Koordinaten d) %
		\coordinate (Bd) at (-90:\ISpitze);
		\coordinate (Ad) at ($(Bd) + (-\Winkel:{\ISpitze/sqrt(2)})$);
		\coordinate (Cd) at ($(Bd) + (0:{\USpitze/sqrt(2)})$);
		% Pfeile d) %
		\draw[Spannungspfeil] (Bd) -- (Cd) node[near end, above] {$\underline{U}$};
		\draw[Strompfeil] (Bd) -- (Ad) node[midway, below] {$\underline{I}$};
		\path (Ad) -- (Bd) -- (Cd) pic[Winkelpfeil2] {angle = Ad--Bd--Cd};
		% Beschriftung d) %
		\node[Beschriftung] at ($(Bd) + (-90:{\ISpitze / \BeschSpace})$) {d)};
		% ---------------------------------------------------------------------- % 
				
	\end{tikzpicture}
	
	\tikzsetnextfilename{Zeigerdiagramm_III_e}
	\begin{tikzpicture}
	%Bild e)....
	
	\end{tikzpicture}
	
	
\end{frame}

%%%%%%%%%%%%%%%%%%%%%%%%%%%%%%%%%%%%%%%%%%%%%%%%%%%%%%%%%%%%%%%%%%%%%%%%%%%%%

\end{document}

