\documentclass[8pt, aspectratio=43]{beamer}

\usepackage{HPE_Lecture}
%\usetikzlibrary{arrows.meta} %In HPE_Lecture.sty eingefügt! (Tikz - Bilder in LaTeX)
\usepackage{ifthen} %usepackage um eine if-Verzweigung zu benutzen
\usepackage{tikz}
\usetikzlibrary{angles, calc, positioning, intersections}

%Booleans um nicht alles immer kompilieren zu müssen...
\newboolean{ZeigerdiagrammI}
\setboolean{ZeigerdiagrammI}{true}
\newboolean{ZeigerdiagrammII}
\setboolean{ZeigerdiagrammII}{false}
\newboolean{ZeigerdiagrammIII}
\setboolean{ZeigerdiagrammIII}{false}


% % % % % % % % % % % % % % % % % % % % % % % % % % %

%Dokument mit allen Ziegerdiagrammen, die im Skript für NuS II enthalten sind. 

% % % % % % % % % % % % % % % % % % % % % % % % % % %

\begin{document}
	
%%%%%%%%%%%%%%%%%%%%%%%%%%%%%%%%%%%%%%%%%%%%%%%%%%%%%%%%%%%%%%%%%%%%%%%%%%%

\begin{frame}\frametitle{Zeigerdiagramm I}

\ifZeigerdiagrammI{
	
\begin{flushright}
		\tikzsetnextfilename{Zeigerdiagramm_I_a}
		\begin{tikzpicture}
			[Spannungspfeil/.style={-{Latex[length=2mm]},thick, blue},						%Definiert den Sytle der Strompfeile
			SinusPlot/.style={domain=0:360, samples=100, variable=\t, thick, color=blue},	%Options for sinus plot	
			Achsenpfeil/.style={->},												  		%Style des Achsenpfeils
			HelpNode/.style={at end, inner sep=0pt, outer sep=0pt, minimum size=0pt},		%Options für Hilfs-Node am Ende der Hilfslinie
			Winkelpfeil/.style={draw, -{Latex[length=1mm]}},								%Style der  Winkelpfeile
			HelpLine/.style={densely dashed, very thin}										%Hilfslinien
			]	
			
			%% Definitionen wie Zeigerlänge etc.. %% 
			\def\USpitze{0.8cm};		%Amplitude von U
			\def\LenSca{200};			%Die Länge des Sinusplot wird: (max_domain - min_domain) / LenSca; in [cm]
			\def\PhiO{0};				%Winkel wt0
			\def\PhiE{30};				%Winkel wt1			
			\def\PhiZ{60};				%Winkel wt2
			\def\TickLen{0.05}			% Halbe Länge des Achsen-Tick
			\def\BeschScale{0.8}		%Skalireung der Achsenbeschrifung
			
			% Definieren von verschiedenen Koordinaten
			\coordinate (xStart) at (-0.2cm,0);		  		  %Startpunkt der x-Achse (wt-Achse)
			\coordinate (xEnd) at (360cm/\LenSca+0.2cm,0); 	  %Endpunkt der x-Achse (wt-Achse)
			\coordinate (yStart) at (0cm,-0.2cm);   		  %Startpunkt der y-Achse
			\coordinate (yEnd) at (0cm,\USpitze+0.2cm);	      %Endpunkt der y-Achse 
			
			%%%%%%%%%%%%%%%%%%%%%%%%%% Zeitbereich %%%%%%%%%%%%%%%%%%%%%%%%%%%%%%%
			% Harmonishce Funktionen plotten: 
			\draw[SinusPlot, name path=u] plot (\t/\LenSca,{\USpitze*sin(\t+\PhiO)});		%Plot u = U*sin(wt+phi0)
			% Achsen Plotten:
			\draw[Achsenpfeil, name path=yAchse] (yStart) -- (yEnd) node[at end, right]{$u$};		 %Plot y-axis
			\draw[Achsenpfeil, name path=xAchse] (xStart) -- (xEnd) node[at end, below]{$\omega$t};  %Plot x-axis
			%Ticks & Beschriftung der x-Achse
			\node[scale = \BeschScale, anchor = north east] at (0/\LenSca,0) {$0$};
			\draw (90/\LenSca,\TickLen) -- (90/\LenSca,-\TickLen) node[scale = \BeschScale, below]{$\frac{\pi}{2}$};		
			\draw (180/\LenSca,\TickLen) -- (180/\LenSca,-\TickLen) node[scale = \BeschScale, below]{$\pi$};
			\draw (270/\LenSca,-\TickLen) -- (270/\LenSca,\TickLen) node[scale = \BeschScale, above]{$\frac{3\pi}{2}$};
			\draw (360/\LenSca,-\TickLen) -- (360/\LenSca,\TickLen) node[scale = \BeschScale, above]{$2\pi$};
			% Bechriftung der Spannungsamplitude:
			\draw[HelpLine] (0,\USpitze) -- (90/\LenSca-\PhiO/\LenSca,\USpitze)
			node[at start, left, color=blue]{$\hat{u}$};
			
			%%%%%%%%%%%%%%%%%%%%%%%%%% Zeigerdiagramm %%%%%%%%%%%%%%%%%%%%%%%%%%%%%%%%%%%
			% Kreise 
			\coordinate (Ursprung) at ($(0,0) - (\USpitze+0.5cm,0)$); 	%Definiert den Ursprung des Zeigerdiagramms
			\draw (Ursprung) circle  (\USpitze);						%Kreis mit Radius von U
			\node[anchor = north east] at (Ursprung) {M};				%Beschriftung des Mittelpunktes
			% Achsen
			\draw ($(Ursprung) - (0,\USpitze+0.2cm)$) -- ($(Ursprung) + (0,\USpitze+0.2cm)$) %Zeichnet y-achse
				node[at start, left, scale = \BeschScale]{$\frac{3\pi}{2}$}					 % 3Pi/2 Beschr.
				node[at end, left, scale = \BeschScale]{{$\frac{\pi}{2}$}};					 % Pi/2 Beschr.
			\draw ($(Ursprung) - (\USpitze+0.1cm,0)$) -- ($(Ursprung) + (\USpitze+0.1cm,0)$) %Zeichnet x-achse
				node[at start, below, scale = \BeschScale]{$\pi$}					 		 % Pi Beschr.
				node[at end, below, scale = \BeschScale]{$0$};					 			 % 0 Beschr.
			% Pfeile 
			\draw[Spannungspfeil] (Ursprung) -- ++(\PhiO:\USpitze) node[near end, below]{$\mzeiger{u}$}
				node(tmp1) [HelpNode]{}; %
			\draw[Spannungspfeil, densely dotted] (Ursprung) -- ++(\PhiE:\USpitze) node(tmp2) [HelpNode]{}; 
			\draw[Spannungspfeil, densely dotted] (Ursprung) -- ++(\PhiZ:\USpitze) node(tmp3) [HelpNode]{}; 
			% Winkel einzeichenen
			% Winkel wt1 einzeichnen:
			\path (xStart) -- (Ursprung) -- (tmp2)
				pic[Winkelpfeil, angle radius = 4mm, pic text=\tiny{$\omega t_1$}, angle eccentricity=1.45] {angle = xStart--Ursprung--tmp2}; 
			% Winkel wt2 einzeichnen:
			\path (xStart) -- (Ursprung) -- (tmp3)
				pic[Winkelpfeil, angle radius = 2.5mm] {angle = xStart--Ursprung--tmp3};	
			\node [anchor = north west] at (Ursprung) {\tiny{$\omega t_2$}}; %Beschriftung für wt2	
			% Hilfslinien
			\draw[HelpLine] (Ursprung |- tmp2) -- (\PhiE/\LenSca,0 |- tmp2) -- (\PhiE/\LenSca, -0.8*\USpitze)
				node[at end, anchor =  east]{$\omega t_1$}; % |- Orthogonal coordinate (manual S.141 (13.3))	
			\draw[HelpLine] (Ursprung |- tmp3) -- (\PhiZ/\LenSca,0 |- tmp3) -- (\PhiZ/\LenSca, -0.8*\USpitze)
				node[at end, anchor =  west]{$\omega t_2$}; % |- Orthogonal coordinate (manual S.141 (13.3))
			% Pfeil für Drehrichtung
			\draw[-{Latex[length=0.7mm]}, very thin] ($(Ursprung) + (0:\USpitze+0.08cm)$) 
				arc[start angle=0, delta angle=20, x radius=\USpitze+0.05, y radius=\USpitze+0.05]
				node[near end, right, scale = \BeschScale]{$\omega t$}; 
			% y-Achsenbeschriftung	
			\draw ($(Ursprung) + (0,0.7*\USpitze)$) -- ++(200:0.15cm) node[at end, anchor = east]{Z};	
		\end{tikzpicture}
	%-----------------------------------------------------------------------------------------------------------------%	

	%-----------------------------------------------------------------------------------------------------------------%		
		\tikzsetnextfilename{Zeigerdiagramm_I_b}
		\begin{tikzpicture}
			[Spannungspfeil/.style={-{Latex[length=2mm]},thick, blue},						%Definiert den Sytle der Strompfeile
			SinusPlot/.style={domain=0:360, samples=100, variable=\t, thick, color=blue},	%Options for sinus plot	
			Achsenpfeil/.style={->},												  		%Style des Achsenpfeils
			HelpNode/.style={at end, inner sep=0pt, outer sep=0pt, minimum size=0pt},		%Options für Hilfs-Node am Ende der Hilfslinie
			Winkelpfeil/.style={draw, -{Latex[length=1mm]}},								%Style der  Winkelpfeile
			HelpLine/.style={densely dashed, very thin}										%Hilfslinien
			]
			
			%% Definitionen wie Zeigerlänge etc.. %% 
			\def\USpitze{0.8cm};		%Amplitude von U
			\def\LenSca{200};			%Die Länge des Sinusplot wird: (max_domain - min_domain) / LenSca; in [cm]
			\def\PhiN{0};				%Winkel 0
			\def\PhiE{45};				%Winkel 1
			\def\PhiZ{90};				%Winkel 2
			\def\PhiD{135};				%Winkel 3
			\def\PhiV{180};				%Winkel 4
			\def\PhiF{225};				%Winkel 5
			\def\PhiS{270};				%Winkel 6
			\def\PhiSi{315};			%Winkel 7
			\def\TickLen{0.05}			% Halbe Länge des Achsen-Tick
			\def\BeschScale{0.8}		%Skalireung der Achsenbeschrifung
			
			%%%%%%%%%%%%%%%%%%%%%%%%%% Zeitbereich waagerecht %%%%%%%%%%%%%%%%%%%%%%%%%%%%%%%
			% Definieren von verschiedenen Koordinaten
			\coordinate (xStart) at (-0.2cm,0);		  		  %Startpunkt der x-Achse (wt-Achse)
			\coordinate (xEnd) at (360cm/\LenSca+0.2cm,0); 	  %Endpunkt der x-Achse (wt-Achse)
			\coordinate (yStart) at (0cm,-\USpitze-0.2cm);    %Startpunkt der y-Achse
			\coordinate (yEnd) at (0cm,\USpitze+0.2cm);	  	  %Endpunkt der y-Achse
			
			% Harmonishce Funktionen plotten: 
			\draw[SinusPlot, name path=u] plot (\t/\LenSca,{\USpitze*sin(\t+\PhiN)});		%Plot u = U*sin(wt+phi0)
			% Achsen Plotten:
			\draw[Achsenpfeil, name path=yAchse] (yStart) -- (yEnd) node[at end, right]{$u$};		 %Plot y-axis
			\draw[Achsenpfeil, name path=xAchse] (xStart) -- (xEnd) node[at end, below]{$\omega$t};  %Plot x-axis
			%Ticks & Beschriftung der x-Achse
			\node[scale = \BeschScale, anchor = north east] at (0/\LenSca,0) {$0$};
			\draw (90/\LenSca,\TickLen) -- (90/\LenSca,-\TickLen) node[scale = \BeschScale, below]{$\frac{\pi}{2}$};		
			\draw (180/\LenSca,\TickLen) -- (180/\LenSca,-\TickLen) node[scale = \BeschScale, below]{$\pi$};
			\draw (270/\LenSca,-\TickLen) -- (270/\LenSca,\TickLen) node[scale = \BeschScale, above]{$\frac{3\pi}{2}$};
			\draw (360/\LenSca,-\TickLen) -- (360/\LenSca,\TickLen) node[scale = \BeschScale, above]{$2\pi$};
			
			%%%%%%%%%%%%%%%%%%%%%%%%%% Zeigerdiagramm %%%%%%%%%%%%%%%%%%%%%%%%%%%%%%%%%%%
			% Kreis und Pfeile 
			\coordinate (Ursprung) at ($(0,0) - (\USpitze+0.5cm,0)$); 	%Definiert den Ursprung des Zeigerdiagramms
			\draw (Ursprung) circle  (\USpitze);						%Kreis mit Radius von U
			% Pfeile
			\foreach \Phi/\tmp/\Besch in 
			{\PhiN/tmp0/0, \PhiE/tmp1/1, \PhiZ/tmp2/2, \PhiD/tmp3/3, \PhiV/tmp4/4, \PhiF/tmp5/5, \PhiS/tmp6/6, \PhiSi/tmp7/7}
			{
				\draw[-{Latex[length=0.8mm]}, very thin] (Ursprung) -- ++(\Phi:\USpitze) node[pos=1.13, scale=\BeschScale]{\Besch} 			 
					node(\tmp)[HelpNode]{}; %	
			}
		
			%%%%%%%%%%%%%%%%%%%%%%%%%% Zeitbereich vertikal %%%%%%%%%%%%%%%%%%%%%%%%%%%%%%%
			% Definieren von verschiedenen Koordinaten für vertikalen cosinus plot
			\coordinate (Nullpunkt2pi) at (-\USpitze-0.5cm,-\USpitze-0.5cm-360cm/\LenSca); %Endpunkt für den Cosinusplot
			\coordinate (Nullpunkt) at (-\USpitze-0.5cm,-\USpitze-0.5cm); 			   	   %Startpunkt für den Cosinusplot
			\coordinate (xStart2) at ($(Nullpunkt) - (\USpitze+0.2cm,0)$);	  			   %Startpunkt der x-Achse 
			\coordinate (xEnd2) at ($(Nullpunkt) + (\USpitze+0.2cm,0)$); 	 	  		   %Endpunkt der x-Achse 
			\coordinate (yStart2) at ($(Nullpunkt) + (0,0.2cm)$);    					   %Startpunkt der y-Achse (wt-Achse)
			\coordinate (yEnd2) at ($(Nullpunkt) - (0,360cm/\LenSca+0.2cm)$);	  	  	   %Endpunkt der y-Achse (wt-Achse)
			
			% Harmonishce Funktionen plotten: 
			\draw[SinusPlot, name path=u2, shift=(Nullpunkt2pi)] plot({\USpitze*cos(\t+\PhiN)},\t/\LenSca);
			% Achsen Plotten:
			\draw[Achsenpfeil, name path=yAchse2] (yStart2) -- (yEnd2) node[at end, right]{$\omega$t}; %Plot y-axis
			\draw[Achsenpfeil, name path=xAchse2] (xStart2) -- (xEnd2);								  %Plot x-axis
			%Ticks & Beschriftung der y-Achse
			\node[scale=\BeschScale, anchor=north east] at (Nullpunkt) {$0$};
			\draw[shift=(Nullpunkt)] (\TickLen,-90/\LenSca) -- (-\TickLen,-90/\LenSca)
				node[scale=\BeschScale, left, fill=white]{$\frac{\pi}{2}$};
			\draw[shift=(Nullpunkt)] (-\TickLen,-180/\LenSca) -- (\TickLen,-180/\LenSca)
				node[scale=\BeschScale, right, fill=white]{$\pi$};
			\draw[shift=(Nullpunkt)] (\TickLen,-270/\LenSca) -- (-\TickLen,-270/\LenSca)
				node[scale=\BeschScale, left, fill=white]{$\frac{3\pi}{2}$};
			\draw[shift=(Nullpunkt)] (\TickLen,-360/\LenSca) -- (-\TickLen,-360/\LenSca)
				node[scale=\BeschScale, left, fill=white]{$2\pi$};	
			%% Hilfslinien %%
			% Hilfslinien für sinus
			\draw[HelpLine] (tmp2) -- (\PhiZ/\LenSca, 0 |- tmp2) node[anchor=south, at end, scale=\BeschScale]{2} -- (\PhiZ/\LenSca, 0);
			\draw[HelpLine] (tmp3) -- (\PhiD/\LenSca, 0 |- tmp3) node[pos=1.05, scale=\BeschScale]{3} -- (\PhiD/\LenSca, 0);
			\draw[HelpLine] (tmp5) -- (\PhiF/\LenSca, 0 |- tmp5) node[pos=0.98, below, scale=\BeschScale]{5} -- (\PhiF/\LenSca, 0);
			\draw[HelpLine] (\PhiF/\LenSca, 0 |- tmp5) -- (\PhiSi/\LenSca, 0 |- tmp7) 
				node[pos=1.2, below, scale=\BeschScale]{7} -- (\PhiSi/\LenSca, 0);
			\draw[HelpLine] (tmp6) -- (\PhiS/\LenSca, 0 |- tmp6) node[at end, below, scale=\BeschScale]{6} -- (\PhiS/\LenSca, 0);
			% Hilfslinien für cosinus
			\draw[HelpLine] (tmp3) -- ($(tmp3 |- 0,-\PhiZ/\LenSca) + (Nullpunkt)$) node[anchor=south, at end, scale=\BeschScale]{3} -- ($(0, \PhiD/\LenSca) + (Nullpunkt)$);
			
				
		\end{tikzpicture}
		
\end{flushright}
}\fi
\end{frame}

%%%%%%%%%%%%%%%%%%%%%%%%%%%%%%%%%%%%%%%%%%%%%%%%%%%%%%%%%%%%%%%%%%%%%%%%%%%

\begin{frame}\frametitle{Zeigerdiagramm II}

\ifZeigerdiagrammII{
	
\begin{flushright}
	\tikzsetnextfilename{Zeigerdiagramm_II_a}
	\begin{tikzpicture}
		%In der folgenden eckigen Klammer können verschiedene Styles definiert werden,
		%welche dann in diesem tikzpicture verwendet werden können. 
		[Strompfeil/.style={-{Latex[length=2mm]},thick, red},							%Definiert den Sytle der Strompfeile
		 SinusPlot/.style={domain=0:360, samples=100, variable=\t, thick, color=red},	%Options for sinus plot	
		 Achsenpfeil/.style={->},												  		%Style des Achsenpfeils
		 HelpNode/.style={at end, inner sep=0pt, outer sep=0pt, minimum size=0pt},		%Options für Hilfs-Node am Ende der Hilfslinie
		 Phasenpfeil/.style={-{Latex[length=0.7mm]}, very thin},				  		%Style des Phasenpfeils (Zeitbereich)
		 Winkelpfeil/.style={pic text=$\varphi_2$, draw, -{Latex[length=1mm]}, angle radius = 3mm, angle eccentricity=1.55},										   					   %Style der  Winkelpfeile
		 HelpLine/.style={densely dashed, very thin}									%Hilfslinien
		 ]
	
		%% Definitionen wie Zeigerlänge und Abstand der Grafiken %%  	
		\def\ISpitzeE{0.9cm};		%Amplitude von I1
		\def\ISpitzeZ{0.65cm};		%Amplitude von I2
		\def\PhiE{0};				%Phasenverschiebung von I1
		\def\PhiZ{50};				%Phasenverschiebung von I2
		\def\LenSca{180};			%Die Länge des Sinusplot wird: (max_domain - min_domain) / LenSca; in [cm]
		\def\TickLen{0.05}			% Halbe Länge des Achsen-Tick
		\def\BeschScale{0.8}		%Skalireung der Achsenbeschrifung
		
		% Definieren von verschiedenen Koordinaten
		\coordinate (xStart) at (-0.2cm,0);		  		  %Startpunkt der x-Achse (wt-Achse)
		\coordinate (xEnd) at (360/\LenSca+0.2,0); 	 	  %Endpunkt der x-Achse (wt-Achse)
		\coordinate (yStart) at (0cm,-\ISpitzeE-0.2cm);   %Startpunkt der y-Achse
		\coordinate (yEnd) at (0cm,\ISpitzeE+0.2cm);	  %Endpunkt der y-Achse 
		
		%%%%%%%%%%%%%%%%%%%%%%%%%% Zeitbereich %%%%%%%%%%%%%%%%%%%%%%%%%%%%%%%
		% Harmonishce Funktionen plotten: 
		\draw[SinusPlot, name path=i1] plot (\t/\LenSca,{\ISpitzeE*sin(\t+\PhiE)});		%Plot i1 = I1*sin(wt+phi1)
		\draw[SinusPlot, name path=i2] plot (\t/\LenSca,{\ISpitzeZ*sin(\t+\PhiZ)});		%Plot i2 = I2*sin(wt+phi2)
		% Achsen Plotten:
		\draw[Achsenpfeil, name path=yAchse] (yStart) -- (yEnd) node[at end, left]{};			 %Plot y-axis
		\draw[Achsenpfeil, name path=xAchse] (xStart) -- (xEnd) node[at end, below]{$\omega$t};  %Plot x-axis
		%Ticks & Beschriftung der x-Achse
		\node[scale = \BeschScale, anchor = north east] at (0/\LenSca,0) {$0$};
		\draw (90/\LenSca,\TickLen) -- (90/\LenSca,-\TickLen) node[scale = \BeschScale, below]{$\frac{\pi}{2}$};		
		\draw (180/\LenSca,\TickLen) -- (180/\LenSca,-\TickLen) node[scale = \BeschScale, below]{$\pi$};
		\draw (270/\LenSca,-\TickLen) -- (270/\LenSca,\TickLen) node[scale = \BeschScale, above]{$\frac{3\pi}{2}$};
		\draw (360/\LenSca,-\TickLen) -- (360/\LenSca,\TickLen) node[scale = \BeschScale, above]{$2\pi$};
		% Pfeil der Phasenverschiebung 
		\draw [name intersections={of=i1 and xAchse, name=interi1}, very thin]
			(interi1-3) -- ++(-90:0.5cm) node(HNode1)[HelpNode]{};	%Plotted Hilfslinie und genereirt Hilfsnode am Ende
		\draw [name intersections={of=i2 and xAchse, name=interi2}, very thin]
			(interi2-2) -- ++(-90:0.5cm) node(HNode2)[HelpNode]{};	%Plotted Hilfslinie und genereirt Hilfsnode am End	
		\draw[Phasenpfeil] ($(0,0.1cm)+(HNode2)$) -- ($(0,0.1cm)+(HNode1)$)
			node[below, near end]{$\varphi_2$};						%Plotted Pfeil mit Beschriftung
			
		%%%%%%%%%%%%%%%%%%%%%%%%%% Zeigerdiagramm %%%%%%%%%%%%%%%%%%%%%%%%%%%%%%%%%%%
		% Kreise 
		\coordinate (Ursprung) at ($(0,0) - (\ISpitzeE+0.5cm,0)$);	%Definiert den Ursprung des Zeigerdiagramms
		\draw ($(Ursprung)+(0,-\TickLen)$) rectangle  ($(Ursprung)+(0,\TickLen)$); %Plotted Mittelpunkt der Kreise
		\draw (Ursprung) circle  (\ISpitzeE);						%Kreis mit Radius von I1
		\draw (Ursprung) circle  (\ISpitzeZ);						%Kreis mit Radius von I2
		% Pfeile 
		\draw[Strompfeil] (Ursprung) -- ++(\PhiE:\ISpitzeE) node[midway, below]{$\mzeiger{i}_1$}
				node(tmp1) [HelpNode]{}; %
		\draw[Strompfeil] (Ursprung) -- ++(\PhiZ:\ISpitzeZ) node[midway, left]{$\mzeiger{i}_2$}
				node(tmp2) [HelpNode]{}; %
		\path (tmp1) -- (Ursprung) -- (tmp2) pic[Winkelpfeil] {angle = tmp1--Ursprung--tmp2}; % Winkel einzeichnen
		% Hilfslinien
		\draw[HelpLine] ($(Ursprung)+(0,\ISpitzeE)$) -- (90/\LenSca-\PhiE/\LenSca,\ISpitzeE) -- ++(-90:\ISpitzeE);
		\draw[HelpLine] ($(Ursprung)+(0,\ISpitzeZ)$) -- (90/\LenSca-\PhiZ/\LenSca,\ISpitzeZ);
		\draw[HelpLine] ($(Ursprung)-(0,\ISpitzeE)$) -- (270/\LenSca-\PhiE/\LenSca,-\ISpitzeE);
		\draw[HelpLine] ($(Ursprung)-(0,\ISpitzeZ)$) -- (270/\LenSca-\PhiZ/\LenSca,-\ISpitzeZ);
		\draw[name intersections={of=i2 and yAchse, name=interi2}, HelpLine] (tmp2) -- (interi2-1);	
		% Beschriftung
		\node (Besch1) [anchor = west] at (140/\LenSca,\ISpitzeE+2.5mm) {$\hat{i}_1\sin(\omega t)$};		 %Beschriftung I1
		\node (Besch2) [anchor = west] at (140/\LenSca,\ISpitzeE-1mm) {$\hat{i}_2\sin(\omega t+\varphi_2)$}; %Beschriftung I2
		\draw (Besch1.west) -- ++ (240:3mm);	%Hilfslinie
		\draw (Besch2.west) -- ++ (240:4mm);	%Hilfslinie		
	\end{tikzpicture}
	%-----------------------------------------------------------------------------------------------------------------%	
	
	%-----------------------------------------------------------------------------------------------------------------%
	\tikzsetnextfilename{Zeigerdiagramm_II_b}
	\begin{tikzpicture}
		%In der folgenden eckigen Klammer können verschiedene Styles definiert werden,
		%welche dann in diesem tikzpicture verwendet werden können. 
		[Strompfeil/.style={-{Latex[length=2mm]},thick, red},							%Definiert den Sytle der Strompfeile
		Spannungspfeil/.style={-{Latex[length=2mm]},thick, blue},						%Definiert den Sytle der Spannungspfeile
		SinusPlot/.style={domain=0:360, samples=100, variable=\t, thick},				%Options for sinus plot	
		Achsenpfeil/.style={->},												  		%Style des Achsenpfeils
		HelpNode/.style={at end, inner sep=0pt, outer sep=0pt, minimum size=0pt},		%Options für Hilfs-Node am Ende der Hilfslinie
		Phasenpfeil/.style={-{Latex[length=0.7mm]}, very thin},				  			%Style des Phasenpfeils (Zeitbereich)
		Winkelpfeil/.style={draw, {Latex[length=1mm]}-, angle radius = 3mm, angle eccentricity=1.5},										   					    %Style der  Winkelpfeile
		HelpLine/.style={densely dashed, very thin}										%Hilfslinien
		]
		
		%% Definitionen wie Zeigerlänge und Abstand der Grafiken %%  	
		\def\USpitze{0.9cm};		%Amplitude von I1
		\def\ISpitze{0.65cm};		%Amplitude von I2
		\def\PhiU{0};				%Phasenverschiebung von I1
		\def\PhiI{-60};				%Phasenverschiebung von I2
		\def\LenSca{180};			%Die Länge des Sinusplot wird: (max_domain - min_domain) / LenSca; in [cm]
		\def\TickLen{0.05}			% Halbe Länge des Achsen-Tick
		\def\BeschScale{0.8}		%Skalireung der Achsenbeschrifung
		
		% Definieren von verschiedenen Koordinaten
		\coordinate (xStart) at (-0.2cm,0);		  		%Startpunkt der x-Achse (wt-Achse)
		\coordinate (xEnd) at (360/\LenSca+0.2,0); 	 	%Endpunkt der x-Achse (wt-Achse)
		\coordinate (yStart) at (0cm,-\USpitze-0.2cm);  %Startpunkt der y-Achse
		\coordinate (yEnd) at (0cm,\USpitze+0.2cm);	  	%Endpunkt der y-Achse 
		
		%%%%%%%%%%%%%%%%%%%%%%%%%% Zeitbereich %%%%%%%%%%%%%%%%%%%%%%%%%%%%%%%
		% Harmonishce Funktionen plotten: 
		\draw[SinusPlot, name path=u, color=blue] plot (\t/\LenSca,{\USpitze*sin(\t+\PhiU)});	%Plot u = U*sin(wt+phi1)
		\draw[SinusPlot, name path=i, color=red] plot (\t/\LenSca,{\ISpitze*sin(\t+\PhiI)});	%Plot i = I*sin(wt+phi2)
		% Achsen Plotten:
		\draw[Achsenpfeil, name path=yAchse] (yStart) -- (yEnd) node[at end, left]{};			 %Plot y-axis
		\draw[Achsenpfeil, name path=xAchse] (xStart) -- (xEnd) node[at end, below]{$\omega$t};  %Plot x-axis
		%Ticks & Beschriftung der x-Achse
		\node[scale = \BeschScale, anchor = north east] at (0/\LenSca,0) {$0$};
		\draw (90/\LenSca,\TickLen) -- (90/\LenSca,-\TickLen) node[scale = \BeschScale, below]{$\frac{\pi}{2}$};		
		\draw (180/\LenSca,-\TickLen) -- (180/\LenSca,\TickLen) node[scale = \BeschScale, above]{$\pi$};
		\draw (270/\LenSca,-\TickLen) -- (270/\LenSca,\TickLen) node[scale = \BeschScale, above]{$\frac{3\pi}{2}$};
		\draw (360/\LenSca,-\TickLen) -- (360/\LenSca,\TickLen) node[scale = \BeschScale, above]{$2\pi$};
		% Pfeile der Phasenverschiebung und Amplituden 
		\draw [name intersections={of=i and xAchse, name=interi}, very thin]
			(interi-1) -- ++(-90:\USpitze) node(HNode1)[HelpNode]{}		%Plotted Hilfslinie und genereirt Hilfsnode am Ende
			(interi-2) -- ++(-90:\USpitze) node(HNode2)[HelpNode]{};	%Plotted Hilfslinie und genereirt Hilfsnode am Ende
		\draw [name intersections={of=u and xAchse, name=interu}, very thin]
			(interu-2) -- ++(-90:\USpitze) node(HNode3)[HelpNode]{};	%Plotted Hilfslinie und genereirt Hilfsnode am End	
		\draw[Phasenpfeil] ($(0,0.1cm)+(HNode3)$) -- ($(0,0.1cm)+(HNode2)$)
			node[below, midway]{$\varphi$};						%Plotted Pfeil mit Beschriftung
		\draw[Phasenpfeil] ($(0,0.1cm)+(HNode1)$) -- (0,-\USpitze+0.1cm)
			node[below, midway]{$\varphi_i$};						%Plotted Pfeil mit Beschriftung
		% Hilfslinien und Amplitudenbeschriftung
		\draw[HelpLine] (0,\USpitze) -- (90/\LenSca-\PhiU/\LenSca,\USpitze)
			node[at start, left, color=blue]{$\hat{u}$};	%Bechriftung der Spannungsamplitude
		\draw[HelpLine] (0,\ISpitze) -- (90/\LenSca-\PhiI/\LenSca,\ISpitze)
			node[at start, left, color=red]{$\hat{i}$};		%Bechriftung der Stromamplitude
		% Y-Achsenbeschriftung
		\node[anchor = west] at (yEnd) {\textcolor{blue}{$u(\omega t)$}, \textcolor{red}{$i(\omega t)$}};
		
		%%%%%%%%%%%%%%%%%%%%%%%%%% Zeigerdiagramm %%%%%%%%%%%%%%%%%%%%%%%%%%%%%%%%%%%
		% Kreise 
		\coordinate (Ursprung) at ($(0,0) - (\USpitze+0.5cm,0)$);	%Definiert den Ursprung des Zeigerdiagramms
		\draw (Ursprung) circle  (\USpitze);						%Kreis mit Radius von U
		\draw (Ursprung) circle  (\ISpitze);						%Kreis mit Radius von I
		% Achsen
		\draw ($(Ursprung) - (0,\USpitze+0.2cm)$) -- ($(Ursprung) + (0,\USpitze+0.2cm)$) %Zeichnet y-achse
			node[at start, left, scale = \BeschScale]{$\frac{3\pi}{2}$}					 % 3Pi/2 Beschr.
			node[at end, left, scale = \BeschScale]{{$\frac{\pi}{2}$}};					 % Pi/2 Beschr.
		\draw ($(Ursprung) - (\USpitze+0.1cm,0)$) -- ($(Ursprung) + (\USpitze+0.1cm,0)$) %Zeichnet x-achse
			node[at start, below, scale = \BeschScale]{$\pi$}					 		 % Pi Beschr.
			node[at end, below, scale = \BeschScale]{$0$};					 		 % 0 Beschr.
		% Pfeile
		\draw[Spannungspfeil] (Ursprung) -- ++(\PhiU:\USpitze) node[midway, above]{$\mzeiger{u}$}
			node(tmp1) [HelpNode]{}; 	%Spannungspfeil mit Hilfsnode am Ende des Zeigers
		\draw[Strompfeil] (Ursprung) -- ++(\PhiI:\ISpitze) node[near end, left]{$\mzeiger{i}$}
			node(tmp2) [HelpNode]{};	%Strompfeil mit Hilfsnode am Ende des Zeigers
		\path (tmp2) -- (Ursprung) -- (tmp1)
		 	pic[draw, {Latex[length=1mm]}-, angle radius = 3mm, angle eccentricity=1.5, pic text=$\varphi_i$]
		 	{angle = tmp2--Ursprung--tmp1} %Pfeil für Winkel Phi_i
		 	pic[draw, arrows={|-{Latex[length=1mm]}}, angle radius = 7mm, angle eccentricity=1.15, pic text=$\varphi$]
		 	{angle = tmp2--Ursprung--tmp1}; %Pfeil für Winkel Phi
		\draw[Phasenpfeil] ($(Ursprung) + (0:\USpitze+0.05cm)$) 
			arc[start angle=0, delta angle=20, x radius=\USpitze+0.05, y radius=\USpitze+0.05]
			node[near end, right, scale = \BeschScale]{$\omega t$}; %Pfeil für Drehrichtung	
	\end{tikzpicture}
		
\end{flushright}
}\fi
\end{frame}

%%%%%%%%%%%%%%%%%%%%%%%%%%%%%%%%%%%%%%%%%%%%%%%%%%%%%%%%%%%%%%%%%%%%%%%%%%%

\begin{frame}\frametitle{Zeigerdiagramm III}

\ifZeigerdiagrammIII{

\begin{flushright}
	
	%-----------------------------------------------------------------------------------------------------------------%
	\tikzsetnextfilename{Zeigerdiagramm_III_a_b_c_d}
	\begin{tikzpicture}
		%In der folgenden eckigen Klammer können verschiedene Styles definiert werden,
		%welche dann in diesem tikzpicture verwendet werden können. 
		[Spannungspfeil/.style={-{Latex[length=2mm]},thick, blue},	%Definiert den Style der Sannungspfeile
		 Strompfeil/.style={-{Latex[length=2mm]},thick, red},		%Definiert den Sytle der Strompfeile
		 Winkelpfeil/.style={pic text=$\varphi$, draw, -{Latex[length=1mm]}, angle radius = 5mm, angle eccentricity=1.3},										   %Style der  Winkelpfeile
		 Winkelpfeil2/.style={pic text=$\varphi$, draw, -{Latex[length=1mm]}, angle radius = 3mm, angle eccentricity=1.5},										   %Style des kleineren Winkelpfeils von Diagramm d) 
		 Beschriftung/.style={anchor=north, gray}					%Style der Diagrammbeschriftungen a), b), ...
		 ]
		
		%% Definitionen wie Zeigerlänge und Abstand der Grafiken %%  
		\def\USpitze{1.3cm};	%Länge des Spannungspfiles
		\def\ISpitze{0.8cm};	%Länge des Strompfeiles
		\def\Winkel{40};		%Winkel zwischen Strom und Spannung
		\def\Xstep{0.3cm};		%Horizontaler Abstand der Grafiken
		\def\BeschSpace{0.25cm};%Vertikalen Abstand der Beschriftung zum waagerechten Pfeil. 
		% ---------------------------------------------------------------------- % 
		%% Diagramm a) %%
		% Koordinaten a) %
		\coordinate (Aa) at (-\Winkel:\ISpitze);	%Koordinaten Definition in Polarform (Winkel:Länge)
		\coordinate (Ba) at (0:0cm);				% Ax ist Spitze des Strompfeils, Bx ist Beginn des Strompfeils
		\coordinate (Ca) at (0:\USpitze);			%Cx ist Spitz des Spannungspfeiles
		% Pfeile a) % 
		\draw[Spannungspfeil] (Ba) -- (Ca) node[midway, above] {$\mzeiger{u}$}; %Spannungspfeil von Bx nach Cx, inkl. Beschriftung
		\draw[Strompfeil] (Ba) -- (Aa) node[midway, below] {$\mzeiger{i}$};		%Strompfeil von Bx nach Ax, inkl. Beschriftung
		\path (Aa) -- (Ba) -- (Ca) pic[Winkelpfeil] {angle = Aa--Ba--Ca};		%Zeichnet den Winkel ein mit Hilfe von pic
		% Beschriftung a) %
		\node[Beschriftung] at (-90:\BeschSpace) {a)};	%Fügt dem Diagramm Beschriftung a) zu. 
		% ---------------------------------------------------------------------- % 
		%% Diagramm b) %% 
		% Koordinaten b) % 
		\coordinate (Bb) at ($(Ca) + (0:\Xstep)$);			%Dank calc library können Koordinaten berechnet werden. 
		\coordinate (Ab) at ($(Bb) + (0:\ISpitze)$);		%Somit können Koordinaten alle relativ berechnet werden
		\coordinate (Cb) at ($(Bb) + (\Winkel:\USpitze)$);
		% Pfeile b) %
		\draw[Spannungspfeil] (Bb) -- (Cb) node[midway, above] {$\mzeiger{u}$};
		\draw[Strompfeil] (Bb) -- (Ab) node[midway, below] {$\mzeiger{i}$};
		\path (Ab) -- (Bb) -- (Cb) pic[Winkelpfeil] {angle = Ab--Bb--Cb};
		% Beschriftung b) %
		\node[Beschriftung] at ($(Bb) + (-90:\BeschSpace)$) {b)};
		% ---------------------------------------------------------------------- % 
		%% Diagramm c) %%
		% Koordinaten c) %
		\coordinate (Bc) at ($(Ab) + (0:\Xstep)$);
		\coordinate (Ac) at ($(Bc) + (0:\ISpitze)$);
		\coordinate (Cc) at ($(Ac) + (180+\Winkel:\USpitze)$); %Cc ist hier der Beginn des Spannungspfeiles
		% Pfeile c) %
		\draw[Spannungspfeil] (Cc) -- (Ac) node[midway, below] {$\mzeiger{u}$};
		\draw[Strompfeil] (Bc) -- (Ac) node[midway, above] {$\mzeiger{i}$};
		\path (Bc) -- (Ac) -- (Cc) pic[Winkelpfeil] {angle = Bc--Ac--Cc};
		% Beschriftung c) %
		\node[Beschriftung] at ($(Bc) + (-90:\BeschSpace)$) {c)};
		% ---------------------------------------------------------------------- % 
		%% Diagramm d) %%
		% Koordinaten d) %
		\coordinate (Bd) at (-90:\ISpitze);
		\coordinate (Ad) at ($(Bd) + (-\Winkel:{\ISpitze/sqrt(2)})$);
		\coordinate (Cd) at ($(Bd) + (0:{\USpitze/sqrt(2)})$);
		% Pfeile d) %
		\draw[Spannungspfeil] (Bd) -- (Cd) node[at end, pos=1.1] {$\underline{U}$};
		\draw[Strompfeil] (Bd) -- (Ad) node[at end, pos=1.1] {$\underline{I}$};
		\path (Ad) -- (Bd) -- (Cd) pic[Winkelpfeil2] {angle = Ad--Bd--Cd};
		% Beschriftung d) %
		\node[Beschriftung] at ($(Bd) + (-90:\BeschSpace)$) {d)};
		% ---------------------------------------------------------------------- % 			
	\end{tikzpicture}
	%-----------------------------------------------------------------------------------------------------------------%	
	
	%-----------------------------------------------------------------------------------------------------------------%	
	\tikzsetnextfilename{Zeigerdiagramm_III_e}
	\begin{tikzpicture}
		%Verschiedene Styles definieren:
		[SinusPlot/.style={domain=-80:360, samples=100, variable=\t, thick},	  %Options for sinus plot
		 Achsenpfeil/.style={->},												  %Style des Achsenpfeils
		 HelpNode/.style={at end, inner sep=0pt, outer sep=0pt, minimum size=0pt},%Options für Hilfs-Node am Ende der Hilfslinie
		 Phasenpfeil/.style={-{Latex[length=0.7mm]}, very thin},				  %Style des Phasenpfeils (Zeitbereich)
		 Spannungspfeil/.style={-{Latex[length=2mm]}, thick},					  %Style der Spannungspfeile
		 Winkelpfeil/.style={draw, -{Latex[length=1mm]}},						  %Style der Winkelpfeile (Zeigerdiagramm)
		 Hilfspfeil/.style={-{Latex[length=1mm]}}								  
		]
		% Verschiedene Variabeln definieren:
		\def\LenSca{140};		%Die Länge des Sinusplot wird: (max_domain - min_domain) / LenSca; in [cm]
		\def\Ueins{1cm};		%Amplitude der Spannung U1	
		\def\Uzwei{0.7cm};		%Amplitude der Spannung U2	
		\def\PhiE{25};			%Phasenverschiebung von U1  
		\def\PhiZ{70};			%Phasenverschiebung von U2  
		\def\Helpa{0.25cm};		%Länge einer Hilfsline (Sinusplot)
		\def\Helpb{0.6cm};		%Länge einer Hilfsline (Sinusplot)
		\def\Helpc{0.9cm};		%Länge einer Hilfsline (Sinusplot)
		\def\ArrowOff{0.5mm};	%Offset des Winkelpfiels zum Ende der Hilfslinie
		\def\yStep{3cm}; 		%Vertikaler Abstand des Sinus- und Zeiger-plots
		\def\xStep{1cm};		%x-Koordinate der y-Achse (Zeigerdiagramm)
		% Definieren von verschiedenen Koordinaten
		\coordinate (xStart) at (-0.6cm,0);		  %Startpunkt der x-Achse (wt-Achse)
		\coordinate (xEnd) at (2.7cm,0); 	 	  %Endpunkt der x-Achse (wt-Achse)
		\coordinate (yStart) at (0cm,-0.9cm); 	  %Startpunkt der y-Achse
		\coordinate (yEnd) at (0cm,\Ueins+\Uzwei);%Endpunkt der y-Achse 
		%%%%%%%%%%%%%%%%%%%%%%%%%% Zeitbereich %%%%%%%%%%%%%%%%%%%%%%%%%%%%%%%
		% Harmonishce Funktionen plotten:
		\draw[SinusPlot, color=blue, name path=u1] plot (\t/\LenSca,{\Ueins*sin(\t+\PhiE)});		%Plot u1 = U1*sin(wt+phi1)
		\draw[SinusPlot, color=red, name path=u2] plot (\t/\LenSca,{\Uzwei*sin(\t+\PhiZ)});			%PLot u2 = U2*sin(wt+phi2)
		\draw[SinusPlot, name path=u] plot (\t/\LenSca,{\Ueins*sin(\t+\PhiE)+\Uzwei*sin(\t+\PhiZ)});%Plot addition u1+u2
		% Achsen Plotten:
		\draw[Achsenpfeil] (yStart) -- (yEnd) node (YachseBesch) [at end, left]{$u$};			 %Plot y-axis
		\draw[Achsenpfeil, name path=xAchse] (xStart) -- (xEnd) node[at end, below]{$\omega$t};  %Plot x-axis
		% Legende einfügen:
		\node (U3wt) [node distance=2mm, right=of YachseBesch] {$u_3(\omega t)$};				%Beschriftung für u(wt)
		\node (U1wt) [on grid, color=blue, node distance=7mm, right=of U3wt] {$u_1(\omega t)$};	%Beschriftung für u1(wt)
		\node (U2wt) [on grid, color=red, node distance=7.5mm, right=of U1wt] {$u_2(\omega t)$};%Beschriftung für u2(wt)
		%% Winkel Beschriftung: %%
		% Phi_1: 
		\draw [name intersections={of=u1 and xAchse}, very thin]
			(intersection-1) -- ++(-90:\Helpa) node(tmp)[HelpNode]{};	%Plotted Hilfslinie
		\draw[Phasenpfeil] ($(0,\ArrowOff)+(tmp)$) -- ($(0,-\Helpa)+(0,\ArrowOff)$)
			node[below, near start, color=blue]{$\varphi_1$};			%Plotted Pfeil mit Beschriftung
		% Phi_u:
		\draw [name intersections={of=u and xAchse}, very thin]
			(intersection-1) -- ++(-90:\Helpb) node(tmp)[HelpNode]{};	%Plotted Hilfslinie
		\draw[Phasenpfeil] ($(0,\ArrowOff)+(tmp)$) -- ($(0,-\Helpb)+(0,\ArrowOff)$)
			node[below, midway]{$\varphi_3$};						 	%Plotted Pfeil mit Beschriftung
		% Phi_2:
		\draw [name intersections={of=u2 and xAchse}, very thin]
			(intersection-1) -- ++(-90:\Helpc) node(tmp)[HelpNode]{};	%Plotted Hilfslinie
		\draw[Phasenpfeil] ($(0,\ArrowOff)+(tmp)$) -- ($(0,-\Helpc)+(0,\ArrowOff)$)
			node[below, midway, color=red]{$\varphi_2$};				%Plotted Pfeil mit Beschriftung
		%%%%%%%%%%%%%%%%%%%%%%%%%%%%%%%%%%%%%%%%%%%%%%%%%%%%%%%%%%%%%%%%%%%%%%%%%%%%%
		%%%%%%%%%%%%%%%%%%%%%%%%%% Zeigerdiagramm %%%%%%%%%%%%%%%%%%%%%%%%%%%%%%%%%%%
		%Definieren verschiedener Koordinaten:
		\coordinate (xStart2) at ($(xStart) - (0,\yStep)$); 	%Startpunkt der x-Achse 
		\coordinate (xEnd2) at ($(xEnd) - (0,\yStep)$);			%Endpunkt der x-Achse
		\coordinate (yStart2) at (\xStep,-\yStep-0.2cm);		%Startpunkt der y-Achse
		\coordinate (yEnd2) at (\xStep,-\yStep+\Ueins+\Uzwei);	%Endpunkt der y-Achse
		% Plotten der Achsen
		\draw[Achsenpfeil, name path=xAchse2] (xStart2) -- (2.7cm,-\yStep);  %Plot x-axis
		\draw[Achsenpfeil, name path=yAchse2] (yStart2) -- (yEnd2); 		 %Plot y-axis
		% Bestimmen des Ursprungs 
		\draw [name intersections={of=xAchse2 and yAchse2, by=Ursprung}] (Ursprung) circle (0pt); %Ursprung bestimmen
		% Plotten der Pfeile 
		\draw[Spannungspfeil, color=blue] (Ursprung) --++ (\PhiE:\Ueins)
			 node[at end, pos=1.2, below, color=blue] {$\underline{\hat{u}}_1$}
			 node(tmp1) [HelpNode]{}; %Spannungspfeil 1 mit Beschriftung und Hilfskoordinate	
		\draw[Spannungspfeil, color=red] (Ursprung) --++ (\PhiZ:\Uzwei)
			node[at end, above, color=red] {$\underline{\hat{u}}_2$}
			node(tmp2) [HelpNode]{}; %Spannungspfeil 2 mit Beschriftung und Hilfskoordinate
		\draw[Spannungspfeil] (Ursprung) --++($(\PhiE:\Ueins) + (\PhiZ:\Uzwei)$)
			node[at end, pos=1.2, below] {$\underline{\hat{u}}_3$}
			node(tmp3) [HelpNode]{}; %Spannungspfeil 3 mit Beschriftung und Hilfskoordinate
		% Plotten der Winkelpfeile
		\path (xEnd2) -- (Ursprung) -- (tmp1)
			 pic[Winkelpfeil, angle radius = 7mm, color=blue]
			 {angle = xEnd2--Ursprung--tmp1};	%Winkelpfiel der Spannung U1
		\node [color=blue] at ($(Ursprung) + (8mm,-1.1mm)$) {$\varphi_1$}; %Beschriftung Phi_1
		\path (xEnd2) -- (Ursprung) -- (tmp3)
			pic[Winkelpfeil, angle radius = 5mm]
			{angle = xEnd2--Ursprung--tmp3};	%Winkelpfiel der Spannung U3	
		\node at ($(Ursprung) + (5.3mm,-1.1mm)$) {$\varphi_3$}; %Beschriftung Phi_3
		\path (xEnd2) -- (Ursprung) -- (tmp2)
			pic[Winkelpfeil, angle radius = 2.5mm, color=red]
			{angle = xEnd2--Ursprung--tmp2}; 	%Winkelpfiel der Spannung U2
		\node [color=red] at ($(Ursprung) + (2.5mm,-1.1mm)$) {$\varphi_2$}; %Beschriftung Phi_2
		% Plotten der Hilfslinien
		\draw[thin, densely dotted]	(tmp1) --++ (\PhiZ:\Uzwei); %Ergänzung zum Parallelogramm
		\draw[thin, densely dotted]	(tmp2) --++ (\PhiE:\Ueins); %Ergänzung zum Parallelogramm
		\draw[color=gray, densely dash dot, thin, name path=h1] (tmp1) --++ (-1.3cm,0cm);	
		\draw[color=gray, densely dash dot, thin, name path=h2] (tmp2) --++ (-1.1cm,0cm);	
		\draw[color=gray, densely dash dot, thin, name path=h3] (tmp3) --++ (-2.5cm,0cm);	
		\draw [name intersections={of=h1 and yAchse2, by=h11}] (h11) circle (0pt); %Bestimmung des Schnittpkt.
		\draw [name intersections={of=h2 and yAchse2, by=h22}] (h22) circle (0pt); %Bestimmung des Schnittpkt.
		\draw [name intersections={of=h3 and yAchse2, by=h33}] (h33) circle (0pt); %Bestimmung des Schnittpkt.
		\draw[Hilfspfeil] ($(Ursprung) - (0.2cm,0cm)$) -- ($(h11) - (0.2cm,0cm)$) 
			node[midway, left, color=blue]{$u_1$}; %Hilfspfeil für U1
		\draw[Hilfspfeil] ($(h22) - (0.2cm,0cm)$) -- ($(h33) - (0.2cm,0cm)$) 
			node[midway, left, color=blue]{$u_1$}; %Hilfspfeil für U1
		\draw[Hilfspfeil] ($(Ursprung) - (0.7cm,0cm)$) -- ($(h22) - (0.7cm,0cm)$) 
			node[midway, left, color=red]{$u_2$};  %Hilfspfeil für U2
		\draw[Hilfspfeil] ($(Ursprung) - (1.2cm,0cm)$) -- ($(h33) - (1.2cm,0cm)$) 
			node[midway, left]{$u_3$};			   %Hilfspfeil für U3
		% Beschiftungen
		\node[anchor=north east, color=gray, align=left, font=\small] at (yEnd2) {Augenblicks-\\werte für $t=0$};
		\node[anchor=north, color=gray] at (xStart2) {e)}; %Plot nummerierung	
	\end{tikzpicture}
	%-----------------------------------------------------------------------------------------------------------------%	
\end{flushright}

}\fi
 	
\end{frame}

%%%%%%%%%%%%%%%%%%%%%%%%%%%%%%%%%%%%%%%%%%%%%%%%%%%%%%%%%%%%%%%%%%%%%%%%%%%%%

\end{document}

